%\usepackage[heading]{ctex}
\ctexset{
	chapter = {
		number = \arabic{chapter},
		format = \heiti\zihao{3}\centering,  % 三号黑体,居中
		aftername = \quad,            % 编号和标题之间的间距
		beforeskip = -1.5ex,          % 调整段前距离
		afterskip = 2.5ex,            % 标题后的垂直间距,用于占2行
		fixskip = true,               % 固定前后间距
		},
	section = {
		format = \heiti\zihao{-3},    % 小三号黑体
		beforeskip = 6pt,             % 段前6磅
		afterskip = 6pt,              % 段后6磅
		aftername = \quad,            % 编号和标题之间的间距
		},
	subsection = {
			format = \heiti\zihao{4},     % 四号黑体
			beforeskip = 3pt,             % 段前3磅
			afterskip = 3pt,              % 段后3磅
			aftername = \quad,            % 编号和标题之间的间距
		}
}
	

\usepackage[normalem]{ulem}	% to use underline \uline{text}
\usepackage{xstring}	%to split long text
\usepackage{fontspec}
\usepackage{setspace}
\usepackage{placeins}

\usepackage{float}      % 浮动体控制
%-----------------------setting font

%\setCJKmainfont{SimSun}

\setCJKmainfont{SimSun}
\setmainfont{Times New Roman}

\newcommand{\zhli}{\CJKfamily{zhli}}

\newcommand{\zhkai}{\CJKfamily{zhkai}}

\newcommand{\zhfs}{\CJKfamily{zhfs}}
%-----------------------setting font


%--------------------------------- set bib style
\usepackage[
	backend=biber,        % 使用biber后端(必须)
	style=gb7714-2015,   % 国标样式(需要安装biblatex-gb7714-2015包)
	sorting=none,         % 按姓名、标题、年份排序
	gbnamefmt=lowercase, % 作者姓名大小写格式
	gbpub=false,         % 不显示"出版地不详"等
	gbalign=left,        % 对齐方式
	doi=false,           % 是否显示DOI
	url=false,           % 是否显示URL
	isbn=false           % 是否显示ISBN
]{biblatex}


\addbibresource{bib/main.bib}  % 假设references.bib和main.tex在同一目录

% 设置参考文献标题格式
\defbibheading{bibliography}[\bibname]{
	\chapter*{#1}
	\addcontentsline{toc}{chapter}{#1}
	\markboth{#1}{#1}
}
%--------------------------------- set bib style




%---------------------------------------------公式配置
\usepackage{amsmath, amssymb, amsfonts, mathtools}        % 核心公式宏包(必加,支持各类公式环境)
% 公式编号格式:章节-序号(如1-1,匹配你的图/表编号规则)
\renewcommand{\theequation}{\arabic{chapter}-\arabic{equation}}

%---------------------------------------------公式配置

% %--------------------------------------------------tex block
\newcommand{\titleZh}{这是一个长长长长长长长中文标题}
%\newcommand{\titleZh}{这是中文标题}



\StrLen{\titleZh}[\len]
\ifnum\len>12
	\newcommand{\titleZhUpper}{\StrMid{\titleZh}{1}{12}}
	\newcommand{\titleZhLower}{\StrGobbleLeft{\titleZh}{12}}
\else
	\newcommand{\titleZhUpper}{\titleZh}
	\newcommand{\titleZhLower}{}
\fi

\newcommand{\titleEN}{This is the English title}

\newcommand{\stuSchool}{机械学院}
\newcommand{\stuMajor}{机械设计及其自动化}
\newcommand{\stuName}{张三}
\newcommand{\stuID}{202511201}
\newcommand{\stuTutor}{李四}
\newcommand{\stuTime}{2026年06月01号}
\newcommand{\headTitle}{华北水利水电大学毕业设计}	% 可以修改成论文


% %--------------------------------------------------tex block


% %--------------------------------------------------geometry
\usepackage[a4paper,
			left=2.5cm,
			right=2.5cm,
			top=2.5cm,
			bottom=2.5cm,
			headheight=14pt,  % 新增:适配小五号字体的页眉高度
			headsep=1.0cm,    % 2.5-1.5;页眉内容底部到正文顶部的距离
			footskip=0.75cm   % 2.5-1.75;页脚内容顶部到正文底部的距离
			]{geometry}

%----------------------------\usepackage{tocloft}


%----------------------------\usepackage{fancyhdr}
\usepackage{fancyhdr}
\pagestyle{fancy}
\fancyhf{}
\fancyhead[C]{\songti\zihao{-5} \headTitle}
\fancyfoot[C]{\thepage}
% 设置页眉下划线宽度,例如0.4pt
\renewcommand{\headrulewidth}{0.4pt}
\renewcommand{\footrulewidth}{0pt}

% 关键新增:重定义plain样式(章节首页默认样式),使其与fancy样式一致
\fancypagestyle{plain}{
	\fancyhf{}  % 清空原有页眉页脚
	\fancyhead[C]{\songti\zihao{-5} \headTitle}	%同时显示章节首页页眉
	\fancyfoot[C]{\thepage}
	\renewcommand{\headrulewidth}{0.4pt}
	\renewcommand{\footrulewidth}{0pt}
}
%----------------------------\usepackage{fancyhdr}



\usepackage{tocloft}

\renewcommand{\contentsname}{{\zihao{3}\bfseries\heiti \hfill 目 \quad 录 \hfill \null}}

% to do: bold


\setlength\cftbeforetoctitleskip{9.5pt} %改变目录页大标题“目录”前间距
\setlength\cftaftertoctitleskip{11pt} %改变目录页大标题“目录”后间距

\renewcommand{\cftchapleader}{\cftdotfill{\cftdot}}	% 
\renewcommand{\cftsecleader}{\cftdotfill{\cftdot}}
\renewcommand{\cftsubsecleader}{\cftdotfill{\cftdot}}


\setlength{\cftbeforechapskip}{0pt}	% to reduce distance between chapter and previous (sub)section

\renewcommand{\cftchapfont}{\heiti\zihao{-4}}	% font for chapter in table of contents
\renewcommand{\cftsecfont}{\zihao{-4}}	% font for sec in table of contents
\renewcommand{\cftsubsecfont}{\zihao{-4}}	% font for sec in table of contents
\setcounter{tocdepth}{3}
%----------------------------\usepackage{tocloft}


%-------------------------------------------set graphicx
\usepackage{graphicx}
\renewcommand{\thefigure}{\arabic{chapter}-\arabic{figure}}	%图1-1,换成横线
\renewcommand{\thetable}{\arabic{chapter}-\arabic{table}}	%表1-1,换成横线



%-------------------------------------------set graphicx


% ===================== 新增:PDF书签/超链接配置 =====================
\usepackage[
colorlinks=true,        % 超链接带颜色(false则为边框)
linkcolor=black,        % 内部链接颜色(如目录跳转到章节)
anchorcolor=black,      % 锚点颜色
citecolor=black,        % 引用颜色
urlcolor=blue,          % 网址颜色
bookmarks=true,         % 启用PDF书签
bookmarksnumbered=true, % 书签显示章节编号(如“1 引言”)
bookmarksopen=true,     % 打开PDF时书签默认展开
bookmarksopenlevel=2,   % 展开层级(2=显示chapter+section)
unicode=true,           % 书签支持中文(关键!避免中文乱码)
pdfencoding=auto       % PDF编码自动适配
]{hyperref}

% 解决ctex与hyperref的中文书签兼容问题(可选,防止特殊情况乱码)
\usepackage{bookmark}      % 增强书签功能,优化中文显示
\bookmarksetup{
	numbered=true,          % 书签带编号
	open=true,              % 展开书签
	level=section           % 展开到section层级
}


% 设置autoref的中文名称
\renewcommand{\figureautorefname}{图}
\renewcommand{\tableautorefname}{表}
\renewcommand{\chapterautorefname}{章}
\renewcommand{\sectionautorefname}{节}
\renewcommand{\subsectionautorefname}{小节}
\renewcommand{\equationautorefname}{公式}
